% planisphere.tex
%
% The LaTeX code in this file brings together into a single document the
% various components of the model planisphere.
%
% Copyright (C) 2014-2024 Dominic Ford <https://dcford.org.uk/>
%
% This code is free software; you can redistribute it and/or modify it under
% the terms of the GNU General Public License as published by the Free Software
% Foundation; either version 2 of the License, or (at your option) any later
% version.
%
% You should have received a copy of the GNU General Public License along with
% this file; if not, write to the Free Software Foundation, Inc., 51 Franklin
% Street, Fifth Floor, Boston, MA  02110-1301, USA

% ----------------------------------------------------------------------------

\documentclass[a4paper,onecolumn,10pt]{article}
\usepackage{xeCJK}  % 支持中日韩字符
\setCJKmainfont{PingFang SC}  % 设置中文字体(宋体)
\usepackage[dvips]{graphicx}
\usepackage{graphicx}
\usepackage{fancyhdr,url}
\usepackage{parskip}
\usepackage[pdftitle={Build your own planisphere}, pdfauthor={Dominic Ford}, pdfsubject={Build your own planisphere}, pdfkeywords={Build your own planisphere}, colorlinks=true, linkcolor=blue, citecolor=blue, filecolor=blue, urlcolor=blue]{hyperref}
\renewcommand{\familydefault}{\sfdefault}
\pagestyle{fancy}

\lhead{\it 制作您的活动星图}
\chead{}
\rhead{\thepage}
\lfoot{}\rfoot{}
\cfoot{\bf\footnotesize\copyright\ 2014--2024 Dominic Ford. 根据GNU通用公共许可证第3版分发。文档下载自 \url{https://in-the-sky.org/planisphere/}}

\fancypagestyle{plain}{%
\fancyhf{} % clear all header and footer fields
\renewcommand{\headrulewidth}{0pt}
\renewcommand{\footrulewidth}{0pt}}

\title{制作您的活动星图}
\author{作者Dominic Ford}
\date{2014--2024}

\begin{document}
\maketitle
\setcounter{footnote}{1}

【“夜语星趴”专用中文版说明:本中文版本tex文本内容、星图标注均使用
Claude 4 Sonnet生成,并经人工编辑,仅适用于浙江高田坑村(北纬30度附近)】

活动星图是一种简单的手持设备,显示在任何特定时间夜空中可见星星的地图。通过旋转星轮,
它展示了星星如何在夜空中移动,以及不同星座在一年中不同时间的可见情况。

在这里,我提供了一个套件,您可以下载并打印,用纸张或硬纸板制作您自己的活动星图。

活动星图的设计取决于使用地点的地理位置,因为从不同地方可以看到不同的星星。我已经为广泛的
纬度范围创建了套件,您可以从以下地址下载:

\url{https://in-the-sky.org/planisphere/}

本文档中提供的活动星图是为纬度\input{tmp/lat}使用而设计的。
 
\section*{您需要的材料}

\begin{itemize}
\item 两张A4纸,或最好是薄卡纸。
\item 剪刀。
\item 一个开脚钉扣件。
\item 可选:一张透明塑料片,例如用于投影仪的醋酸纤维片。
\item 可选:少量胶水。
\end{itemize}

\section*{组装说明}

{\bf 步骤 1} -- 活动星图的外观因您居住的地方而略有不同。本文档中准备的活动星图适用于
地球上纬度\input{tmp/lat}几度范围内的任何地方。如果您居住在别处,您应该从以下地址下载
其他套件:

\url{https://in-the-sky.org/planisphere/}

{\bf 步骤 2} -- 将此PDF文件后面显示星轮和活动星图主体的页面打印到两张不同的纸上,
或最好打印到薄卡纸上。

{\bf 步骤 3} -- 小心地剪出星轮和活动星图的主体。同时剪出活动星图主体的灰色阴影区域,
如果您有的话,还要剪出打印在透明塑料上的网格线。如果您使用硬纸板,您可能希望沿着虚线
小心地在活动星图主体上刻痕,以便稍后更容易沿此线折叠。

{\bf 步骤 4} -- 星轮中心有一个小圆圈,活动星图主体底部有一个匹配的小圆圈。在每个上面
打一个小孔(直径约2毫米)。如果手头有打孔器,那是理想的,否则使用圆规尖端并通过圆周
运动来扩大孔洞。

{\bf 步骤 5} -- 将开脚钉扣件穿过星轮中心,扣件头部紧贴星轮的印刷面。然后将活动星图
主体套在同一个扣件上,印刷面朝向扣件背面。将扣件向下折叠以固定两张硬纸板。

{\bf 步骤 6 (可选)} -- 如果您将PDF文件的最后一页打印在塑料片上,现在应该将这个网格线
粘贴到您从活动星图主体上剪出的观测窗口上。

{\bf 步骤 7} -- 沿着虚线折叠活动星图主体,使星轮的正面通过您在主体上剪出的窗口显示出来。

{\bf 恭喜,您的活动星图现在可以使用了!}

\section*{如何使用您的活动星图}

转动星轮,直到找到边缘标有今日日期的位置,将此位置与当前时间对齐。观测窗口现在显示
天空中所有可见的星座。

走到室外并面向北方。将活动星图举向天空,观测窗口底部标记的星星应与您在前方天空中看到的
星星相匹配。

转身面向东方或西方,旋转活动星图,使"东"或"西"字样位于窗口底部。再次确认,观测窗口
底部的星星应与您在前方天空中看到的星星相匹配。

如果您将高度和方位角网格线打印在透明塑料上,这些线条可以让您计算出天体在天空中的高度
和方向。圆圈绘制在地平线上方10、20、30、...、80度的高度上。作为参考,十度的距离大约
相当于手臂伸直时的一个手掌宽度。曲线是连接地平线上的点到您头顶正上方点的垂直线。
它们绘制在基本方向S、SSE、SE、ESE、E等。

\section*{定制活动星图}

这个活动星图套件是使用一系列Python脚本和pycairo图形库设计的。如果您想定制您的活动星图,
欢迎您从我的GitHub账户下载脚本并修改它们,但请注明来源:

\url{https://github.com/dcf21/planisphere}

\section*{许可证}

与{\tt In-The-Sky.org}上的其他内容一样,这些活动星图套件的版权归\copyright\ Dominic Ford所有。
但是,{\tt In-The-Sky.org}上的所有内容都是为了全世界业余天文学家的利益而提供的,您可以在
以下条件下修改和/或重新分发本网站上的任何材料:(1) 任何有相关版权文本的项目{\bf 必须}在您的
重新分发版本中包含该{\bf 未修改}的文本,(2) 您{\bf 必须}注明我,Dominic Ford,为原作者和
版权持有人,(3) 您{\bf 不得}从您对本网站材料的复制中获得任何利润,{\bf 除非}您是明确以
推进天文科学为目标的注册慈善机构,{\bf 或者}您获得了作者的书面许可。

\newpage

\centerline{\includegraphics{tmp/starwheel}}

\vspace{1cm}
活动星图的中央星轮,应夹在折叠的支架内。

\newpage
\thispagestyle{empty}
\vspace*{-3.0cm}
\centerline{\includegraphics{tmp/holder}}
\newpage

\centerline{\includegraphics{tmp/altaz}}

\vspace{1cm}
这个网格线可以选择性地打印在透明塑料上,并粘贴到活动星图主体的剪出窗口中,
以显示天空中天体的高度和方向。

\end{document}

